%-------------------------
% Resume in Latex
% Author : Thomas Ottaway adapted from Sourabh Bajaj's Template
% License : MIT
%------------------------

\documentclass[letterpaper,11pt]{article}

\usepackage{latexsym}
\usepackage[empty]{fullpage}
\usepackage{titlesec}
\usepackage{marvosym}
\usepackage[usenames,dvipsnames]{color}
\usepackage{verbatim}
\usepackage{enumitem}
\usepackage[pdftex]{hyperref}
\usepackage{hyperref}
\hypersetup{
    colorlinks=true,
    linkcolor=blue,
    filecolor=magenta,      
    urlcolor=cyan, }
\usepackage{fancyhdr}


\pagestyle{fancy}
\fancyhf{} % clear all header and footer fields
\fancyfoot{}
\renewcommand{\headrulewidth}{0pt}
\renewcommand{\footrulewidth}{0pt}

% Adjust margins
\addtolength{\oddsidemargin}{-0.375in}
\addtolength{\evensidemargin}{-0.375in}
\addtolength{\textwidth}{1in}
\addtolength{\topmargin}{-.5in}
\addtolength{\textheight}{1.0in}

\urlstyle{same}

\raggedbottom
\raggedright
\setlength{\tabcolsep}{0in}

% Sections formatting
\titleformat{\section}{
  \vspace{-4pt}\scshape\raggedright\large
}{}{0em}{}[\color{black}\titlerule \vspace{-5pt}]

%-------------------------
% Custom commands
\newcommand{\resumeItem}[2]{
  \item\small{
    \textbf{#1}{#2 \vspace{-5pt}}
  }
}

\newcommand{\resumeSubheading}[4]{
    \vspace{3pt} 
    \begin{tabular*}{0.97\textwidth}{l@{\extracolsep{\fill}}r}
      \textbf{#1} & #2 \\
      \textit{\small#3} & \textit{\small #4} \\
    \end{tabular*}\vspace{-3pt}
}

\newcommand{\resumeSubheadingSimple}[2]{
  \vspace{-1pt}\item
    \begin{tabular*}{0.97\textwidth}{l@{\extracolsep{\fill}}r}
      \textbf{#1} & \textit{\small#2}\\
    \end{tabular*}\vspace{-3pt}
}

\newcommand{\resumeSubItem}[2]{\resumeItem{#1}{#2}\vspace{-5pt}}

\renewcommand{\labelitemii}{$\circ$}

% \newcommand{\resumeSubHeadingListStart}{\begin{itemize}[leftmargin=*]}
% \newcommand{\resumeSubHeadingListEnd}{\end{itemize}}
\newcommand{\resumeSubHeadingListStart}{}
\newcommand{\resumeSubHeadingListEnd}{}
\newcommand{\resumeItemListStart}{\begin{itemize}}
\newcommand{\resumeItemListEnd}{\end{itemize}}

%-------------------------------------------
%%%%%%  CV STARTS HERE  %%%%%%%%%%%%%%%%%%%%%%%%%%%%


\begin{document}

%----------HEADING-----------------
\begin{tabular*}{\textwidth}{l@{\extracolsep{\fill}}r}
  \text{\Large {Thomas Ottaway}}  & tottaway123@gmail.com $\cdot$ (518) 466--9711 $\cdot$ \href{https://github.com/tottaway}{Github} \\
\end{tabular*}


%-----------Paid Experience---------------
\section{Work Experience}
  \resumeSubHeadingListStart
  \resumeSubheading{Argo AI (C++)}{Pittsburgh, PA}
    {Software Engineer}{Jul 2022 -- Oct 2022}
      \resumeItemListStart
        \resumeItem{}
      {Worked on complementary perception stack (CAVS) which provided emergency braking capabilities to the AV}
        \resumeItem{}
      {Enabled the use of high resolution lidar data on limited compute by reducing perception algorithm runtime by $>$40\%}
        \resumeItem{}
      {Improved redundancy throughout the CAVS system, making safety critical signals less susceptible to data corruption}
        \resumeItem{}
      {Led software development for end-to-end testing suite which would allow the CAVS system to be enabled in fleet}
        \resumeItem{}
      {Helped develop new visualization framework which reduced overhead time to review logs from 15 min to $<$1 min}
      \resumeItemListEnd
  \resumeSubheading{Neocis Inc (C++)}{Miami, FL}
    {Software Engineering Intern}{May 2021 -- Aug 2021}
      \resumeItemListStart
        \resumeItem{}
      {Developed algorithm to automatically transfer pre-planned surgeries onto pre-op CT scans}
        \resumeItem{}
      {Developed algorithm to detect and classify splints in CT scans}
        \resumeItem{}
      {Developed novel haptics to facilitate more accurate drilling at steep angles and in hard to reach places}
      \resumeItemListEnd
  \resumeSubheading{Draper Laboratories (Python, C++)}
    {Cambridge, MA}{Engineering Intern}{Jun 2020 -- Aug 2020}
      \resumeItemListStart
        \resumeItem{}
        {Developed and implemented a search algorithm using a particle filter and information-theoretic search heuristics}
        \resumeItem{}
        {Applied search algorithm to locate and classify radioactive and gaseous hazards using a mobile sensor} 
        \resumeItem{}
        {Implemented improved simulations for gaseous plumes}
      \resumeItemListEnd
      \resumeSubheading{NYS DOH/Health Research Institute (Python, SQL, JavaScript)}
    {Albany, NY}{Software Developer Intern}{Jul 2018 -- Jun 2020}
      \resumeItemListStart
        \resumeItem{}
        {Migrated an application storing the location of COVID-19 test samples from an Oracle back end to SQL Server}
        \resumeItem{}
        {Created systems to import patient identifying information as CSVs which drastically reduced data entry times}
        \resumeItem{}
        {Developed Python scripts to pre-process laboratory instrument data and perform automated regression testing}
        % \resumeItem{}
        % {Interfaced with a legacy database containing 200+ tables}
        % \resumeItem{}
        % {Built automated systems for HL7 message generation to expand Remote Order Entry accessibility}
      \resumeItemListEnd
    % \resumeSubheading{LingView (JavaScript, Node.js, React)}{Providence, RI}{Developer}{Sep 2018 -- May 2019}
      % \resumeItemListStart
        % \resumeItem{}
        % {Built a front-end interface for viewing and searching through a linguistic corpus}
        % \resumeItem{}
        % {Collaborated with other students and faculty to determine project goals and priorities}
        % \resumeItem{}
        % {Explored various search technologies such as Fuse, LUNR, and SOLR}
      % \resumeItemListEnd
  \resumeSubHeadingListEnd

%-----------EDUCATION-----------------
\section{Education}
  \resumeSubHeadingListStart
    \resumeSubheading
        {Brown University}{Providence, RI}
        {Applied Math--Computer Science $\cdot$ GPA: 4.0}{2018 -- 2022}
        \resumeItemListStart
            \resumeItem{Technical Coursework: }
            {Operating Systems $\cdot$ Sublinear Algorithms for Big Data $\cdot$ Advanced 3D Perception $\cdot$ Dynamical Systems $\cdot$ Distributed Systems $\cdot$ Convex Optimization $\cdot$ Information Theory $\cdot$ Computer Networks}
            \resumeItem{Teaching Assistant: } {Discrete Math (Spring 2020, Spring 2021, Spring 2022)}
        \resumeItemListEnd
   
    % \resumeSubheadingSimple{Albany High School}{Albany, NY}

    % \resumeSubheadingSimple
    %   {Independent Learning}{}
    %   \resumeItemListStart
    %     \resumeItem{}
    %       {Independent reading of computer vision research papers (primarily ResNets and variations such as Wide ResNets, Stochastic Depth, Pyramid nets etc.)}
    %   \resumeItemListEnd
  \resumeSubHeadingListEnd



%-----Projects and Volunteer Work-------
\section{Projects}
\vspace{3pt}
\small{
\begin{description}
  \item[Fetch] (\textit{Spring 2022}) -- Developed perception, planning, and control pipelines on a Boston Dynamics Spot robot to perform object search and retrieval (\href{https://youtu.be/IvIS-P1sRSk}{video})
  \item[TCP/IP] (\textit{Spring 2022}) -- Implemented a TCP/IP stack in Rust
  \item[Path Tracer] (\textit{Fall 2021}) -- Built path tracer and 3D physics engine in C++ and rendered a \href{https://youtu.be/wIA1xijYGs0}{video} of a galton board
  \item[OS Kernel] (\textit{Spring 2021}) -- Built a kernel in C with a scheduler, TTY, S5FS file system, and virtual memory
  \item[Ball Balancing] (\textit{Spring 2021}) -- Built a simulation of a table with two axis of rotation balancing a ball. Implemented a physics engine, simulated raytraced camera, image processing pipeline, Kalman filter, and PID controller (\href{https://youtu.be/TIc4otyw-zQ}{video})
  % \item[Music Visualizer] (\textit{May 2020}) -- Built a music visualizer in Python using Fourier transforms.
  % \item[Neural Nets for Solving ODEs] (\textit{Aug 2019}) -- Implemented algorithms from \href{https://arxiv.org/abs/physics/9705023}{this paper} in Python.
            % \resumeItem{}
            % {Implementing custom optimization algorithm with quadratic convergance}
  \item[Rocket Stabilization] (\textit{Spring 2019}) -- Studied bifurcations in the dynamics of a TVC rocket as control parameters varied

  % \item[Dynamic System Visualizer] (\textit{Apr 2019}) -- Created visualizations for complex non-linear systems such as the 2-D wave equation and the Lorenz Attractor using Matplotlib and PyQtGraph.
        %   \resumeItemListEnd
        % \resumeSubheadingSimple
        %   {Exploring Numerical Methods}{Oct. 2018 -- Apr. 2019}
        %   \resumeItemListStart
        %     \resumeItem{}
        %       {Explored algorithms for finding numerical approximations to differential equations using Python's numpy library}
        %     \resumeItem{}
        %       {Created visualizations for systems such as the 3-D wave equation and particles in chaotic systems}
        %   \resumeItemListEnd
\end{description}
}
% -------Skills--------
\section{Skills}
  \resumeSubHeadingListStart
    \item \textbf{Programming Languages} -- C++, C, Python, JavaScript, Go, SQL, MATLAB
    % \item \textbf{Communication \& Adaptibility}: Ability to enter a team of diverse people, listen closely, and work collaboratively to push a project forward
    \item \textbf{Tools} -- Eigen, VTK/ITK, Numpy, Matplotlib, ROS, Oracle, PyTorch, protobuf, zmq
    \item \textbf{Algorithms} -- Particle Filters, Kalman Filters, Mophological Operations, 3D Registration, Mask R-CNN, Random Forests
  \resumeSubHeadingListEnd
\section{Activities}
    \resumeSubHeadingListStart
    \item Juggling, running, rock climbing, cooking, blues dancing, unicycling, bicycling, playing piano
  \resumeSubHeadingListEnd
%-------------------------------------------
\end{document}
